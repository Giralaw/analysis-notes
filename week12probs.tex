\documentclass[11pt]{article} 

\usepackage{graphicx}
\usepackage{amsmath,amssymb,amsthm}
\usepackage{tikz}
\usepackage{textcomp}

\usepackage{mathtools}
\DeclarePairedDelimiter\ceil{\lceil}{\rceil}
\DeclarePairedDelimiter\floor{\lfloor}{\rfloor}

\newcommand{\doctitle}{ }
\makeatletter
\renewcommand{\ps@headings}{%
\renewcommand{\@evenhead}{\parbox{\textwidth}{\hrulefill 
\fbox{\sffamily Math 101 -- Real Analysis 2}\hrulefill } }%
\renewcommand{\@oddhead}{\parbox{\textwidth}{\hrulefill 
\fbox{\sffamily Math 101 -- Real Analysis 2}\hrulefill}}}
\makeatother


\linespread{.9}
\hoffset=-0in    \voffset=-.5in
\oddsidemargin=0in   \evensidemargin=0in
\topmargin=-.25in
\textwidth=6.5in   \textheight=9.5in
\columnseprule=.3pt  
\setlength{\parskip}{1em}



\usepackage{multirow}
\usepackage{multicol} 

\newcommand{\mytitle}[1]{{
\begin{centering}
{\Large \sffamily Math 101 -- Real Analysis 2}\\
\bigskip\bigskip{\Large \sffamily \bfseries{#1}}\\
\bigskip\bigskip\end{centering}
}}

\newcommand{\mytitlecompact}[1]{{

\hfill
{\Large \sffamily \bfseries{#1}}
\hfill
%\bigskip
}}


\newcommand{\mysection}[1]{{
\smallskip
\noindent
{\large \sffamily \bfseries{#1}}
}}


\newcommand{\mysubsection}[1]{{
\smallskip
{\sffamily \bfseries{#1}}
}}

\pagestyle{empty}
\pagestyle{headings}

\def\N{{\mathbb{N}}}
\def\Z{{\mathbb{Z}}}
\def\Q{{\mathbb{Q}}}
\def\R{{\mathbb{R}}}
\def\C{{\mathbb{C}}}

\newcommand{\solbox}[1]{\framebox{\parbox{\dimexpr\linewidth-2\fboxsep-2\fboxrule}{#1}}}

\begin{document}

\mytitlecompact{Week 12 Problems}
\nolinebreak \begin{center} Alex Skeldon \end{center}

\begin{enumerate}
  \item (Bass 2.3) We give a counterexample. Let $X = \N$. Let $\mathcal{A}_n$ be the $\sigma$-algebra consisting of all sets of natural numbers less than or equal to $n$, as well as the complements of each of those sets, and $\emptyset$ and $\N$. Then for every set $A$ in the algebra, we can say that either $A$ or $A^c$ is finite. For $i \in \N$, every singleton set $\lbrace 2i \rbrace$ lies in at least one of $\mathcal{A}_n$, but the infinite countable union of all such sets, which is all of the evens, does not lie in any of the $\mathcal{A}_n$ collections, so our nested union fails the property to be a $\sigma$-algebra.
  
\item (H1.4) Let the Lebesgue measure be defined as usual, with the set $X = \R$. By Theorem 4.16 in Bass there exists a subset $E \subset \R$ which is non-measurable. By Theorem 4.6, the collection of measurable sets under an outer measure is a $\sigma$-algebra, so the complement $E^c$ must also be non-measurable. The union of these two sets is the entire set $\R$, which \textit{is} measurable due to the aforementioned $\sigma$-algebra property.

\item (H3) Let $A \subset \R$. If $A$ is open, then for any point $x \in A$, there is an open interval (an open ball in $\R$) around $x$ of some size $\varepsilon > 0$ that is contained in $A$. This ball $(x - \varepsilon, x + \varepsilon)$ has lebesgue measure $2 \varepsilon > 0$, and since it is a subset of $A$, then by property (2) of outer measures, $m(A) \geq 2\varepsilon > 0$ (this includes the case where $m(A) = \infty$, which we consider larger than 0 for inequality purposes).
\newline \text{\qquad} If $A$ is compact, then by the Heine-Borel theorem, $A$ is closed and bounded. Since $A$ is bounded, there is some $n \in \R$ such that $A \subset [-n,+n]$. The measure of this bounding interval is $2n < \infty$, and by the same subset inequality property of outer measures, $m(A) \leq 2n < \infty$.
\end{enumerate}

\end{document}
