\begin{sloppypar}
\section{Seminar 12: Presentations}
\indent Hunter initial remarks: Comparison to reals from OH. Definition of irrational numbers from Cauchy sequences or Dedekind cuts. We are thinking hard about the definition of Lebesgue integration.

Patrick starts presentation of Chapter Two. Hunter asks about the intersection of collections of nothing. Consider a sum of integers. What should the empty sum be? 0, because we're talking about unions. The product of no things should be 1. So the empty union should be the empty set, and the empty intersection should be the WHOLE SET.

Syntax for countable vs. uncountable - countable are with the n = 1 and infinity upper bound. Uncountable are something like $a \in A$. An index set $I$ that we're using can be anywhere from the empty set to an uncountably infinite set.

Jay presents proofs of parts 3 and 4 of proposition 3.5. Jay was wondering why a complete measure space is called complete. He also was wondering why sigma finite measure spaces are called such. Hunter says just old language, and $\sigma$ is an indicator for countability.

Denny and Spencer cover chapter 4. Spencer ends with Cantor set discussion. The Cantor set is uncountable, which is rather unintuitive, all the elements in the cantor set are boundary points, and it contains no intervals (those two things go together).

Hunter: Later, when we integrate over any measure, we will call it the Lebesgue integral, even when it's not over the Lebesgue measure (?!).

Hunter asks Jay to give the description of the bijection from the Cantor set to the unit interval (Hunter: it's like the fibonacci sequence - there are lots of things that are true about it): basically each cantor element is mapped to a number with 0's and 2's, because anything with a 1 gets removed.

PROBLEMS TIME

\end{sloppypar}
