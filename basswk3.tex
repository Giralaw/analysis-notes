\begin{sloppypar}

\setcounter{chapter}{7}
\chapter{Properties of Lebesgue Integrals}
Propositions about functions being zero a.e. and approximations.

\section{criteria for a zero a.e. function}
\begin{prop}
  Suppose $f$ is measurable and non-negative and $\int f \,d\mu = 0$. Then $f = 0$ almost everywhere.
\end{prop}
\begin{proof}
  If $f$ is not equal to $0$ almost everywhere, there exists an $n$ such that $\mu(A_n) > 0$ where $A_n = \{x : f(x) > \frac{1}{n}\} $. But since $f$ is non-negative,
  \[
    0 = \int f \ge \int_{A_n}f \ge \frac{1}{n}\mu(A_n),\]
    a contradiction.
\end{proof}

\begin{prop} 
Suppose f is real-valued and integrable and for every measurable set $A$ we have $\int_A f \,d\mu = 0$. Then $f=0$ a.e.
\end{prop}

\begin{corollary}
  Let $m$ be Lebesgue measure and $a \in \R$. Suppose $ f:  \R \to \R $ is integrable and $\int_a^x f(y) \,dy = 0$ for all $x$. Then $f = 0$ a.e.
\end{corollary}

\section{An approximation result}

\begin{theorem}
  Suppose $f$ is a Lebesgue measurable real-valued integrable function on $\R$. Let $\varepsilon > 0$. Then there exists a continuous function $g$ with compact support such that
  \[
  \int|f-g| < \varepsilon
  .\] 
\end{theorem}

\chapter{Riemann integrals}
In which we show that the Riemann integral of a function exists iff the set of discontinuities of the function of the function have Lebesgue measure zero; in this case, the two integrals agree.

\section{Comparison with the Lebesgue integral}
Recall the Riemann integral definition from Munkres. We only consider bounded functions from $[a,b]$ into $\R$. Here we denote Riemann integrals as $R(f)$.
\[
  \overline{R}(f) = \inf \{U(P,f) : P \text{ is a partition} \} 
.\] and
\[
  \underline{R}(f) = \sup \{L(P,f) : P \text{ is a partition} \} 
.\]
The Riemann integral exists if the two are equal.

\begin{theorem}
  A bounded real-valued function $f$ on $[a,b]$ is Riemann integrable iff the set of points at which $f$ is discontinuous has Lebesgue measure $0$, and in that case, $f$ is Lebesgue measurable and the Riemann integral of $f$ is equal in value to the Lebesgue integral of $f$.
\end{theorem}

\begin{eg}
  Let $[a,b] = [0,1]$ and $f=\chi_A$, where $A$ is the set of irrational numbers in $[0,1]$. If $x \in [0,1]$, every neighborhood of $x$ contains both rational and irrational points, so $f$ is continuous at no point of $[0,1]$. Therefore, $f$ is not Riemann integrable.
\end{eg}

\begin{eg}
  Define $f(x)$ on $[0,1]$ to be $0$ if $x$ is irrational and to be $\frac{1}{q}$ if $x$ is rational and equals $\frac{p}{q}$ when in reduced form. $f$ is discontinuous at every rational. If $x$ is irrational and $ \varepsilon > 0$, there are only finitely many rationals $r$ for which $f(r) \ge \varepsilon$,so taking $\delta$ less than the distance from $x$ to any of this finite collection of rationals shows that $|f(y) - f(x)| < \varepsilon$ if $|y-x| < \delta$. Hence $f$ is continuous at $x$. Therefore the set of discontinuities is a countable set, hence of measure $0$, hence $f$ is Riemann integrable.
\end{eg}

\chapter{Types of convergence}
There are various ways in which a sequence of functions $f_n$ can converge, and we compare some of them. Assume all functions in this chapter to be measurable.

\section{Definitions and examples}
\begin{definition}
  If $\mu$ is a measure, we say a sequence of measurable functions $f_n$ \textit{converges almost everywhere} to $f$ and write $f_n \to f$ a.e. if there is a set of measure $0$ such that for $x$ not in this set we have $f_n(x) \to f(x)$.
  \newline We say $f_n$ \textit{converges in measure} to $f$ if for each $ \varepsilon > 0$ 
  \[
  \mu( \{x : |f_n(x) - f(x)| > \varepsilon\} ) \to 0
.\] as $n \to \infty$.
\newline Let $1 \le p < \infty$. We say $f_n$ \textit{converges in $L^p$ } to $f$ if
  \[
  \int|f_n - f|^p \,d\mu \to 0
  .\] as $n \to \infty$.
\end{definition}

\begin{definition}
  ($L^p$ norms definition from chapter 15) Let ($X, \mathcal{A},\mu)$ be a $\sigma$-finite measure space. For $1 \le p < \infty$, define the $L^p$ norm of $f$ by
  \[
    \|f\|_{p} = (\int |f(x)|^p \,d\mu)^{\frac{1}{p}} 
  .\] 
For $p = \infty$, define the $L^\infty$ norm of $f$ by

\[
\|f\|_\infty = \inf \{M \ge 0 : \mu( \{x : |f(x)| \ge  M\}) = 0\} .
\]
If no such $M$ exists, then $||f||_\infty = \infty$. Thus the $L^\infty$ norm of a function $f$ is the smallest number $M$ such that $|f| \le M$ a.e.
\end{definition}

\begin{prop}
  (1) Suppose $\mu$ is a finite measure. If $f_n \to f$ a.e., then $f_n$ converges to $f$ in measure.
  (2) If $\mu$ is a measure, not necessarily, finite, and $f_n \to f$ in measure, there is a subsequence $n_j$ such that $f_{n_j} \to f$ a.e.
\end{prop}
\begin{proof}
  (1) Let $ \varepsilon > 0$ and suppose $f_n \to f$ a.e. If
  \[
  A_n = \{x : |f_n(x) - f(x)| > \varepsilon\} ,\] 
  then $\chi_{A_n} \to 0$ a.e., and by the dominated convergence theorem,
  \[
    \mu(A_n) = \int \chi_{A_n}(x) \mu(dx) \to 0
  .\] This proves (1).
\end{proof}

\begin{eg}
  Part (1) of the above proposition is not true if $\mu(X) = \infty$. To see this, let $X = \R$ and let $f_n = \chi_{(n,n+1)}$. We have $f_n \to 0$ a.e., but $f_n$ does not converge in measure.
\end{eg}

The next proposition compares convergence in $L^p$ to convergence in measure. Before we prove this, we prove an easy preliminary result know as Chebyshev's inequality.

\begin{lemma}
  If $1 \le p < \infty$, then
  \[
    \mu( \{x : |f(x)| \ge a\}) \le \frac{\int|f|^p d\mu}{a^p}
  .\] 
\end{lemma}
\begin{proof}
  Let $A = \{x : |f(x)| \ge a\} .$ Since $\chi_A \le |f|^p \chi_A / a^p$, we have
  \[
    \mu(A) \le \int_A \frac{|f|^p}{a^p} \,d\mu \le \frac{1}{a^p}\int|f|^p \,d\mu
  .\] This is what we wanted.
\end{proof}

\begin{prop}
  If $f_n$ converges to $f$ in $L^p$, then it converges in measure.
\end{prop}
\begin{proof}
  If $ \varepsilon > 0$, by Chebyshev's inequality
  \[
    \mu( \{x : |f_n(x) - f(x)| > \varepsilon\} ) \le \frac{\int|f_n - f|^p}{ \varepsilon^p} \to 0
  .\] as required.
\end{proof}

\begin{eg}
  Let $f_n = n^2 \chi_{(0,\frac{1}{n})}$ on $[0,1]$ and let $\mu$ be Lebesgue measure. This gives an example where $f_n$ converges to $0$ a.e. and in measure, but does not converge in $L^p$ for any $p \ge 1$.
\end{eg}

\begin{theorem}
  (Egorov's Theorem) Suppose $\mu$ is a finite measure, $ \varepsilon > 0$, and $f_n \to f$ a.e. Then there exists a measurable set $A$ such that $\mu(A) < \varepsilon$ and $f_n \to f$ uniformly on $A^c$.
\end{theorem}
This type of convergence is sometimes known as \textit{almost uniform convergence}. Egorov's theorem is not as useful for solving problems as one might expect, and students have a tendency to try to use it when other methods work much better.
\begin{proof}
  Let
\[
  A_{nk} = \cup_{m=n}^{\infty} \{x : |f_m(x) - f(x)| > \frac{1}{k}\}
.\] 
For fixed $k$, $A_{nk}$ decreases as $n$ increases. The intersection $\cap_n A_{nk}$ has measure $0$ because for almost every $x$, $|f_m(x) - f(x) | \le  \frac{1}{k}$ if $m$ is sufficiently large. Therefore $\mu(A_{nk}) \to 0$ as $n \to \infty$. We can thus find an integer $n_k$ such that $\mu(A_{n_k k}) < \varepsilon_2^{-k}$. Let
\[
  A = \cup_{k=1}^{\infty}A_{n_k k}
.\] Hence $\mu(A) < \varepsilon$. If $x \not\in A$, then $x \not\in  A_{n_k k}$, and so $|f_n(x) - f(x) | \le \frac{1}{k}$ if $n \ge n_k$. Thus $f_n \to f$ uniformly on $A^c$.
\end{proof}

\end{sloppypar}
