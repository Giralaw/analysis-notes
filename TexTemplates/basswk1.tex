\begin{sloppypar}
\chapter{Preliminaries}
\section{Notation and terminology}
We use $A^c$ for the set of points not in A. Specifically, define
$$A^c = \lbrace x \in X : x \notin A \rbrace . $$
We write
$$A - B = A \cap B^c $$ 
(it is common to also see $A \backslash B$) and
$$A \triangle B = (A - B) \cup (B-A) $$
\indent The set $A \triangle B$ is called the \textit{symmetric difference} of $A$ and $B$ and is the set of points that are in one of the sets but not the other. If $I$ is some non-empty index set, a collection of subsets $\lbrace A_{\alpha} \rbrace_{\alpha \in I}$ is disjoint if $A_\alpha \cap A_\beta = \emptyset$ whenever $\alpha \neq \beta$.
\newline \indent We write $A_i \uparrow$ if $A_1 \subset A_2 \subset \cdots$ and write $A_i \uparrow A$ if in addition $A = \cup^{\infty}_{i=1} A_i$. Similarly, $A_i \downarrow$ means $A_1 \supset A_2 \supset \cdots$ and $A_i \downarrow A$ means that in addition $A = \cap^{\infty}_{i=1} A_i$.
\newline \indent We use $\log x$ to denote the natural logarithm of $x$, that is, the logarithm of $x$ to the base $e$.
\newline \indent We use $\Q$ to refer to the set of rational nembers, $\R$ the set of real numbers, and $\C$ the set of complex numbers. We use
$$x \vee y = \max(x,y) \qquad \text{and} x \wedge y = \min(x,y)$$
We can write a real number $x$ in terms of its positive and negative parts: $$x = x^+ \vee 0 \qquad \text{and} \qquad x^- = (-x) \vee 0.$$
If $z$ is a complex number, then $\overline{z}$ is the complex conjugate of $z$. The composition of two functions is defined by $f \circ g(x) = f(g(x))$.
\newline \indent If $f$ is a function whose domain is the reals or a subset of the reals, then $f(x+) = \lim_{y \rightarrow x+} f(y)$ and $f(x-) = \lim_{y \rightarrow x-} f(y)$ are the right and left hand limits of $f$ at $x$, resp.
\newline \indent We say a function $f: \R \rightarrow \R$ is \textit{increasing} if $x < y$ implies $f(x) \leq f(y)$ and $f$ is \textit{strictly increasing} if $x < y$ implies $f(x) < f(y)$. Decreasing and strictly decreasing are defined similarly. A function is monotone if $f$ is either increasing or decreasing.
\newline Given a sequence $\lbrace a_n \rbrace$ of real numbers,
$$\lim\sup_{x\to\infty} a_n = \inf_n \sup_{m \geq n} a_m,$$
$$\lim\inf_{x\to\infty} a_n = \sup_n \inf_{m \geq n} a_m.$$
For example, if
\begin{equation}
a_n=
    \begin{cases}
      1, & \text{n even;}\\
      -1/n, & \text{n odd,}
    \end{cases}
\end{equation}
then $\lim \sup_{n\to\infty}a_n = 1$ and $\lim\inf_{n\to\infty} = 0$. The sequence $\lbrace a_n \rbrace$ has a limit if and only if $\lim \sup_{n\to\infty}a_n = \lim\inf_{n\to\infty}a_n$ and both are finite. We use analogous definitions when we take a limit along the real numbers. For example,
$$\lim\sup_{y\to x}f(y) = \inf_{\delta >0} \sup_{|y-x|< \delta}f(y).$$

\section{Some undergraduate math}

We recall some definitions and facts from undergraduate topology, algebra, and analysis. Many proofs are omitted.
\bigskip

\indent A set $X$ is a \textit{metric space} if there exists a function $d: X \times X \to \R$, called the metric, such that 
\newline (1) $d(x,y) = d(y,x)$ for all $x,y \in X$;
\newline (2) $d(x,y) \geq 0$ for all $x,y \in X$ and $d(x,y) = 0$ iff $x = y$;
\newline (3) $d(x,z) \leq d(x,y) + d(y,z)$ for all $x,y,z \in X$.
\newline \indent Condition (3) is called the triangle inequality.
\medskip
\newline \indent Given a metric space $X$, let
$$B(x,r) = \lbrace y \in X: d(x,y) < r \rbrace$$
be the \textit{open ball} of radius $r$ centered at $x$.
If $A \subset X$, the \textit{interior} of $A$, denoted $A^o$, is the set of $x$ such that there exists $r_x > 0$ with $B(x,r_x) \subset A$.
The closure of $A$, denoted $\overline{A}$, is the set of $x \in X$ such that every open ball centered at $x$ contains at least one point of $A$.
A set $A$ is open if $A = A^o$, closed if $A = \overline{A}$.
If $f: X \to \R$, the support of $f$ is the closure of the set $\lbrace x: f(x) \neq 0 \rbrace$. $f$ is continuous at a point $x$ if given $\varepsilon > 0$, there exists $\delta > 0$ such that $|f(x) - f(y)| < \varepsilon$ whenever $d(x,y) < \delta$.
$f$ is continuous if it is continuous at every point of its domain.
\newline \indent Recall also the topological definition of continuity, convergence of sequences, Cauchy sequences, and completeness of a set $X$.
\newline \indent An open cover of a subset $K$ of $X$ is a non-empty collection $\lbrace G_\alpha \rbrace_{\alpha \in I}$ of open sets such that $K \subset \cup_{\alpha \in I}G_{\alpha}$. The index set $I$ can be finite or infinite. A set $K$ is compact if every open cover contains a finite subcover, i.e. there exists $G_1,\cdots,G_n \in \lbrace G_\alpha \rbrace_{\alpha in I}$ such that $K \subset \cup^n_{i=1}G_i$.
\pagebreak
\newline \textbf{Proposition 1.1} \textit{If $K$ is compact, $F \subset K$, and $F$ is closed, then $F$ is compact}
\medskip \newline \textbf{Proposition 1.2} \textit{If $K$ is compact and $f$ is continuous on $K$, then there exist $x_1$ and $x_2$ such that $f(x_1) = \inf_{x \in K} f(x)$ and $f(x_2) = \sup_{x \in K} f(x)$. In other words, $f$ takes on its maximum and minimum values.}
\medskip \newline \textbf{Remark 1.3} If $x \neq y$, let $r = d(x,y)$ and note that $B(x,r/2)$ and $B(y,r/2)$ are disjoint open sets containing $x$ and $y$, resp. Therefore metric spaces are also what are called hausdorff spaces.
\bigskip \newline \indent The eight vector space properties (AKA linear space), normed linear space when map $x \to ||x||$ satisfying three props, metric induced by the norm, equivalence relationships.
\newline \indent Given an equivalence relationship, $X$ can be written as the union of disjoint equivalence classes. $x$ and $y$ are in the same equivalence class if and only if $x \sim y$.
\bigskip \newline \indent A set $X$ has a partial order "$\leq$" if
\newline (1) $x \leq x$ for all $x \in X$;
\newline (2) if $x \leq y$ and $y \leq x$, then $x = y$;
\newline (3) if $x \leq y$ and $y \leq z$, then $x \leq z$.
\medskip \newline \indent Note that given $x,y \in X$, it is not necessarily true that $x \leq y$ or $y \leq x$. For an example, let $Y$ be a set, let $X$ be the collection of all subsets of $Y$, and say $A \leq B$ if $A,B \in X$ and $A \subset B$.
\medskip \newline \indent We need the following three facts about the real line.
\bigskip \newline \textbf{Proposition 1.4} \textit{Suppose $K \in \R$, $K$ is closed, and $K$ is contained in a finite interval. Then $K$ is compact.}
\bigskip \newline \textbf{Proposition 1.5} \textit{Suppose $G \subset \R$ is open. Then $G$ can be written as the countable union of disjoint open intervals.}
\bigskip \newline \textbf{Proposition 1.6} \textit{Suppose $f: \R \to \R$ is an increasing function. Then both $\lim_{y\to x+}f(y)$ and $\lim_{y\to x-} f(y)$ exist for every $x$. Moreover, the set of $x$ where $f$ is not continuous is countable.}

\section{Proofs of Propositions}

Skipped.

\chapter{Families of Sets}
\section{Algebras and $\sigma$-algebras}

When we turn to constructing measures in Chapter 4, we will see that we cannot in general define the measure of an arbitrary set. The class of sets that we will want to use are $\sigma$-algebras.
\medskip \newline \indent Let $X$ be a set.

\bigskip \textbf{Definition 2.1} An algebra is a collection $\mathcal{A}$ of subsets of $X$ such that
\newline (1) $\emptyset \in  \mathcal{A}$ and $X \in \mathcal{A}$;
\newline (2) if $A \in \mathcal{A}$, then $A^c \in \mathcal{A}$;
\newline (3) if $A_1,\cdots,A_n \in \mathcal{A}$, then $\cup^n_{i=1}A_i = \cap^n_{i=1}A_i$ are in $\mathcal{A}$.

\medskip \noindent $\mathcal{A}$ is a $\sigma$-$algebra$ if in addition
\newline (4) whenever $A_1,A_2,\cdots$ are in $\mathcal{A}$, then $\cup^\infty_{i=1}A_i$ and $\cap^\infty_{i=1}A_i$ are in $\mathcal{A}$.

\bigskip
\noindent In (4) we allow countable unions and intersections only. The intersection requirement is redundant since it is the complement of an infinite union of complementary sets.

\medskip
\noindent The pair ($X,\mathcal{A})$ is called a measurable space. A set $A$ is measurable or $\mathcal{A}$ measurable if $A \in \mathcal{A}$.

\medskip \noindent $\textbf{Example 2.2}$ Let $X = \R$, the set of real numbers, and let $\mathcal{A}$ be the collection of all subsets of $\R$. Then $\mathcal{A}$ is a $\sigma$-algebra.

\medskip \noindent \textbf{Lemma 2.7} \textit{If $\mathcal{A}_\alpha$ is a $\sigma$-algebra for each $\alpha$ in some non-empty index set $I$, then $\cap_{\alpha \in I}\mathcal{A}_\alpha$ is a $\sigma$-algebra.}

If we have a collection $\mathcal{C}$ of subsets of $X$,  define
$$\sigma(\mathcal{C}) = \cap \lbrace \mathcal{A}_\alpha : \mathcal{A}_\alpha \text{ is a } \sigma \text{-algebra, } \mathcal{C} \in \mathcal{A}_\alpha \rbrace ,$$
\newline We call $\sigma(\mathcal{C})$ the $\sigma$-algebra generated by the collection $\mathcal{C}$. Since $\sigma(C)$ is a sigma algebra, then $\sigma(\sigma(C)) = \sigma(C)$.
\newline \indent If $X$ has some additional structure (metric space), then we can talk about open sets. If $\mathcal{G}$ is the collection of open subsets of $X$, then we call $\sigma{\mathcal{G}}$ the Borel $\sigma$-algebra on $X$, and denote this $\mathcal{B}$. Elements of that are called Borel sets, said to be Borel measurable. We will see later that if $X=\R$, then $\mathcal{B} \neq$ the collection of all subsets of $X$.

\noindent \textbf{Proposition 2.8} \textit{If $X = \R$, then the Borel $\sigma$-algebra is generated by each of the following collections of sets:}
  \newline (1) $\mathcal{C}_1 = \lbrace (a,b): a,b \in \R \rbrace $;
  \newline (2) $ \mathcal{C}_2 =  \lbrace [a,b]:a,b \in \R \rbrace$;
  \newline (3) $\mathcal{C}_3 = \lbrace (a,b] : a, b \in \R \rbrace ;$
  \newline (4) $\mathcal{C}_4 = \lbrace (a,\infty) : a \in \R \rbrace ;$
  \section{The monotone class theorem}
  \noindent \textbf{Definition 2.9} A monotone class is a collection of subsets $\mathcal{M}$ of $X$ such that
\newline (1) if $A_i \uparrow A$ and each $A_i \in \mathcal{M}$, then $A \in \mathcal{M}$
\newline (2) if $A_i \downarrow A$ and each $A_i \in \mathcal{M}$, then $A \in \mathcal{M}$.
\newline The intersection of monotone classes is a monotone class, and the intersection of all monotone classes containing a given collection of sets is the smallest monotone class containg that collection.

\noindent \textbf{Theorem 2.10 (The Monotone Class Theorem)} \textit{Suppose $\mathcal{A}_0$ is an algebra, $\mathcal{A}$ is the smallest $\sigma$-algebra containing $\mathcal{A}_0$, and $\mathcal{M}$ is the smallest monotone class containing $\mathcal{A}_0$. Then $\mathcal{M} = \mathcal{A}.$}

\chapter{Measures}
\indent In which we generalize length, Additivity property extended to countable unions (but not uncountable)

\noindent \textbf{Definition 3.1} Let $X$ be a set and $\mathcal{A}$ a sigma algebra consisting of subsets of $X$. A measure on $(X,\mathcal{A})$ is a function $\mu : \mathcal{A} \rightarrow [0,\infty]$ such that
\newline (1) $\mu(\emptyset) = 0$;
\newline (2) if $A_i \in \mathcal{A}, i =1,2,\cdots,$ are pairwise disjoint, then
$$\mu(\cup_{i=1}^{\infty}A_i)=\sum_{i=1}^{\infty}\mu(A_i).$$

Definition 3.1(2) is known as countable additivity. We say a set
function is finitely additive if the measure of the union commutes equals the sum of the measures for pairwise disjoint sets.
The triple $(X, \mathcal{A}, \mu)$ is called a measure space.
Examples include the counting measure, assigning positive values to every member of the reals and then taking the sum of the elements in a sets' values under that assignment, and point mass.
\bigskip
\newline \textbf{Proposition 3.5} The following hold:
\newline (1) If $A,B \in \mathcal{A}$ with $A \subset B,$ then  $\mu(A) \leq \mu(B)$
\newline (2) If $A_i \in \mathcal{A}$ and $A = \cup_{i=1}^\infty A_i$, then $\mu(A) \leq \sum_{i=1}^\infty \mu(A_i).$
\newline (3) Suppose $A_i \in \mathcal{A}$ and $A_i \uparrow A$. Then $\mu(A) = \lim_{n \rightarrow \infty} \mu(A_n).$
\newline (4) Suppose $A_i \in \mathcal{A}$ and $A_i \downarrow A$. If $\mu(A_1) < \infty$, then we have $\mu(A) = \lim_{n \to \infty}\mu(A_n).$

\noindent \textbf{Definition 3.7} A measure $\mu$ is a finite measure if $\mu(X) < \infty$. A measure $\mu$ is sigma finite if there exist sets $E_i \in \mathcal{A}$ for $i=1,2,\ldots$ such that $\mu(E_i) < \infty$ for each $i$ and $X = \cup_{i=1}^\infty E_i$. If $\mu$ is a finite measure, then $X,\mathcal{A},\mu)$ is called a finite measure space, and similarly if $\sigma$-finite measure, then it's called a $\sigma$-finite measure space.

Let $(X, \mathcal{A}, \mu)$ be a measure space. Call $A \subset X$ a null set if it is a subset of some set $B \in \mathcal{A}$ such that $\mu(B) = 0$.
\newline A measure space is complete if all are null sets are in $\mathcal{A}$.
The completion of $\mathcal{A}$ is $(X, \overline{\mathcal{A}}, \overline{\mu})$ such that $\overline{\mathcal{A}}$ is the smallest $\sigma$-algebra such that there is an extension $\overline{\mu}$ of $\mu$ such that $X, \overline{\mathcal{A}},\overline{\mu}$ is complete.
That is, $\overline{\mu}(B) = \mu(B)$ if $B \in \mathcal{A}$.

A probability measure is a measure $\mu$ such that $\mu(X) = 1$.

\chapter{Construction of measures}
Here we address a method for constructing measures, in the process introducing outer measures, the one-dimensional Lebesgue measure, results and examples related to that, and the inability to define Lebesgue measure for all subsets of the reals; finally the Carath\'eodory extension theorem.
\section{Outer measures}
\noindent \textbf{Definition 4.1} Let $X$ be a set. An outer measure is a function $\mu^*$ defined on the collection of all subsets of $X$ satisfying
\newline (1) $\mu^*(\emptyset) = 0$;
\newline (2) if $A \subset B$, then $\mu^*(A) \leq \mu^*(B)$;
\newline (3) $\mu^*(\cup_{i=1}^\infty A_i) \leq \sum_{i=1}^\infty \mu^*(A_i)$ whenever $A_1,A_2,\ldots$ are subsets of $X$.

\noindent \textbf{Definition 4.5} Let $\mu^*$ be an outer measure. A set $A \subset X$ is $\mu^*-measurable$ if
$$\mu^*(E) = \mu^*(E \cap A) + \mu^*(E \cap A^c)$$
for all $E \subset X$

\noindent \textbf{Theorem 4.6 (Carath\'eodory criterion/theorem - NOT EXTENSION)}
\newline \textit{If $\mu^*$ is an outer measure on X, then the collection $\mathcal{A}$ of $\mu^*$-measurable sets is a $\sigma$-algebra. If $\mu$ is the restriction of $\mu^*$ to $\mathcal{A}$, then $\mu$ is a measure. Morever, $\mathcal{A}$ contains all the null sets.}

\setcounter{section}{3}
\section{The Carath\'eodory extension theorem}
\noindent \textbf{Theorem 4.17} Suppose $\mathcal{A}_0$ is an algebra and $l: \mathcal{A}_0 \to [0,\infty]$ is a measure on $\mathcal{A}_0$. Define
$$\mu^*(E) = \inf \lbrace \sum_{i=1}^\infty l(A_i): \textit{each } A_i \in \mathcal{A}_0, E \subset \cup_{i=1}^\infty A_i \rbrace$$
For $E \subset X$. Then
\newline (1) $\mu^*$ is an outer measure;
\newline (2) $\mu^*(A) = l(A)$ if $A \in \mathcal{A}_0$;
\newline (3) Every set in $\mathcal{A}_0$ and every $\mu^*$-null set is $\mu^*$-measurable;
\newline (4) if $l$ is $\sigma$-finite, then there is a unique extension to $\sigma(\mathcal{A}_0)$.

\end{sloppypar}
